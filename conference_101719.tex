\documentclass[conference]{IEEEtran}
\IEEEoverridecommandlockouts
% The preceding line is only needed to identify funding in the first footnote. If that is unneeded, please comment it out.
\usepackage{cite}
\usepackage{amsmath,amssymb,amsfonts}
\usepackage{algorithmic}
\usepackage{graphicx}
\usepackage{hyperref}
\usepackage{textcomp}
\usepackage{xcolor}
\def\BibTeX{{\rm B\kern-.05em{\sc i\kern-.025em b}\kern-.08em
    T\kern-.1667em\lower.7ex\hbox{E}\kern-.125emX}}

\makeatletter
\newcommand{\linebreakand}{%
  \end{@IEEEauthorhalign}
  \hfill\mbox{}\par
  \mbox{}\hfill\begin{@IEEEauthorhalign}
}
\makeatother

\begin{document}

\title{Maneuvering Machine Learning Algorithms to presage the attacks of \textit{Fusarium oxysporum} on Cotton Leaves\\}

\author{\IEEEauthorblockN{1\textsuperscript{st} Anurag Dutta}
\IEEEauthorblockA{\textit{Department of Computer Science and Engineering} \\
\textit{Government College of Engineering \& Textile Technology}\\
Serampore, Calcutta, India \\
anuragdutta.research@gmail.com}
\and 
\IEEEauthorblockN{2\textsuperscript{nd} Pijush Kanti Kumar}
\IEEEauthorblockA{\textit{Department of Information Technology} \\
\textit{Government College of Engineering \& Textile Technology}\\
Serampore, Calcutta, India \\
pijush752000@yahoo.com}
}

\maketitle

\begin{abstract}
Web technologies have reached unprecedented levels during this time of modernization. Significant and relevant technological stacks like as IoT (Internet of Things), ML (Machine Learning), and AI influenced crawling and cradling (Artificial Intelligence). These categories are really beneficial. In this work, we would try to make use of the notion of Machine Learning Algorithms to predict the attack of \textit{Fusarium oxysporum} on the leaves of the Cotton plant. It's a type of ascomycete fungi that forms an infrageneric grouping called section. All of the species, variations, and forms discovered by Wollenweber and Reinking are Elegans. It belongs to the \textit{Nectriaceae} family. Many strains of the F. oxysporum complex are soil-borne plant pathogens, especially in agricultural settings, despite the fact that their primary function in native soils may be as benign or even advantageous plant endophytes or soil saprophytes. There are many textile products that are made from cotton. Cotton is used in a variety of products besides the textile industry, including gill nets, coffee filters, tarpaulins, cotton paper, and bookbinding. Cotton used to be used to make fire hoses. With yearly production of roughly 18.53 million tonnes and 17.14 million tonnes, respectively, India and China are the leading producers of cotton as of 2017. The majority of this production is used by their respective textile industries. This contributes a major portion of the economy. To strengthen the same, we can make use of certain prediction technique that could clearly foresee if the leaves of cotton suffering from the attack by the pathogens, making use of algorithms like 'Support Vector Machine', 'Random Forest', 'k - Nearest Neighbours', and many more. Further this work would also compare the efficacy of these algorithms in predicting the damage in the Cotton Leaves. All Codes, Data, and Supplementary Material are made available at \href{https://github.com/Anurag-Dutta/Maneuvering-Machine-Learning-Algorithms-to-presage-the-attacks-of-Fusarium-oxysporum-on-Cotton-Leave}{https://github.com/Anurag-Dutta/Maneuvering-Machine-Learning-Algorithms-to-presage-the-attacks-of-Fusarium-oxysporum-on-Cotton-Leave}\\
\end{abstract}

\begin{IEEEkeywords}
Machine Learning, Textile Industry, Cotton, Fusarium oxysporum.
\end{IEEEkeywords}

\end{document}
